\chapter{PENDAHULUAN}

Sistem Pembayaran Online Pajak Daerah (SPOPD) dalam komunikasinya dengan tempat pembayaran menggunakan dua metode yaitu :

\begin{enumerate}[1.]
  \item Dengan mengikuti standar spesifikasi ISO-8583, yang berlaku hanya untuk pembayaran tunggal untuk tiap Nomor Objek Pajak (NOP).
  
  \item Dengan format data JSON (\textit{JavaScript Object Notation}) yang mampu melakukan transaksi untuk beberapa Nomor Objek Pajak (NOP) sekaligus.
\end{enumerate}

\begin{enumerate}[A.]

\item \textbf{Spesifikasi ISO-8583}

Pada standar spesifikasi ISO-8583, SPOPD menggunakan beberapa jenis paket data untuk mengakomodasi jenis-jenis transaksi yang dilayani. Adapun jenis transaksi yang dilayani oleh SPOPD adalah pembayaran Surat Pemberitahuan Pajak Terhutang (SPPT) Pajak Bumi dan Bangunan Perdesaan dan Perkotaan.

Klasifikasi jenis paket data yang digunakan dalam SPOPD adalah : 

\begin{itemize}
  \item \textit{Financial Transaction}
  \item \textit{Reversal}
  \item \textit{Network Management}
\end{itemize}

Detail jenis-jenis paket data yang digunakan dalam SPOPD dapat dilihat pada tabel berikut. 

\begin{tabular}{| l | l | c | c | c |}
  \hline
  KLASIFIKASI & JENIS PAKET DATA & KODE PENGENAL & \textit{INCOMING} & \textit{OUTGOING} \\
  \hline  
\end{tabular}

\item \textbf{Format Data JSON}

Model komunikasi dengan format data JSON akan dibahas pada bagian tersendiri dalam dokumen ini.

\end{enumerate}