\chapter{JENIS-JENIS PAKET DATA}

\begin{enumerate}[A.]

  \item \textbf{\textit{Financial Transaction}}
  
  Gambaran mengenai data elemen yang disertakan dalam paket data SPOPD untuk jenis paket data \textit{Financial Transaction Request} dapat dilihat pada tabel \ref{tab:fin-trans}.
  
\begin{table}[H]
  \centering
  \scriptsize
  \begin{tabular}{|p{3em}|p{10em}|p{3em}|p{3em}|l|l|}
  \hline
  \multicolumn{2}{|c|}{\textbf{Data Element}} & \multicolumn{2}{|p{6em}|}{\textbf{Klasifikasi Paket Data}} & \multirow{2}{*}{\textbf{Format}} & \multirow{2}{*}{\textbf{Attr}} \\
  \cline{1-4}
  \textbf{Bit Map} & \textbf{Deskripsi} & \textbf{Req} & \textbf{Res} &  & \\
  \hline
  \hline
  & Message Type & 0200 & 0210 & & \\
  \hline
  & Primary Bit Map & M & M & & B64 \\
  \hline
  P-2 & Primary Account Number & M & M & LLVAR & N..16 \\
  \hline
  P-3 & Processing Code & M & M & & N6 \\
  \hline
  P-4 & Transaction Amount & M & M & & N12 \\
  \hline
  P-7 & Transmission Date and Time & M & M & MMDDhhmmss & N10 \\
  \hline
  P-11 & System Trace Audit Number & M & M & & N6 \\
  \hline
  P-12 & Local Transaction Time & M & M & hhmmss & N6 \\
  \hline
  P-13 & Local Transaction Date & M & M & MMDD & N4 \\
  \hline
  P-18 & Merchant Type (Kode Produksi) & M & M & & N4 \\
  \hline
  P-32 & Acquiring Institution Identification Code & M & M & LLVAR & N4 \\
  \hline
  P-37 & Retrieval Reference Number & M & M & & AN12 \\
  \hline
  P-39 & Response Code & & M & & N2 \\
  \hline
  P-41 & Card Acceptor Terminal Identification & M & M & & AN16 \\
  \hline
  P-48 & Additional Data & M & M & LLLVAR & ANS..210 \\
  \hline
  P-49 & Transaction Currency Code & M & M & & N3 \\
  \hline
  P-63 & Reserved Private & M & M & LLLVAR & N..2 \\
  \hline
  \end{tabular}
  \caption{Tabel Referensi Untuk \textit{Financial Transaction}}
  \label{tab:fin-trans}
\end{table}
  
  \item \textbf{\textit{Reversal Message}}
  
  Gambaran mengenai data elemen yang disertakan dalam paket data SPOPD untuk jenis paket data \textit{Reversal} dapat dilihat pada tabel \ref{tab:reversal}
  
  \begin{table}[H]
    \centering
    \scriptsize
    \begin{tabular}{|p{3em}|p{10em}|p{3em}|p{3em}|l|l|}
    \hline
    \multicolumn{2}{|p{13em}|}{\textbf{Data Element}} & \multicolumn{2}{|p{6em}|}{\textbf{Klasifikasi Paket Data}} & \multirow{2}{*}{\textbf{Format}} & \multirow{2}{*}{\textbf{Attr}} \\
    \cline{1-4}
    \textbf{Bit Map} & \textbf{Deskripsi} & \textbf{Req} & \textbf{Res} & & \\
    \hline
    \hline
    & Message Type & 0400 / 0401 & 0410 / 0411 & & \\
    \hline
    & Primary Bit Map & M & M & & B64 \\
    \hline
    \end{tabular}
    \caption{Tabel Referensi Untuk Paket Data \textit{Reversal}}
    \label{tab:reversal}
  \end{table}
  
  \item \textbf{\textit{Network Management Message}}

\end{enumerate}

Sedangkan untuk paket data yang dikirimkan oleh Sistem Bank/Tempat Pembayaran ke SPOPD memiliki kejadian alur paket data \textit{request}, \textit{reversal}, dan \textit{timeout} seperti berikut ini :

\begin{itemize}

  \item \textbf{\textit{Paket Data Request}}
  
  \item \textbf{\textit{Paket Data Reversal}}
  
  \item \textbf{\textit{Timeout}}
  
  \item \textbf{\textit{Paket Data Network Management}}
  
  \begin{itemize}
    \item \textbf{\textit{Logon/echo-test dikirimkan oleh Sistem Bank/TP}}
    \item \textbf{\textit{Logoff dikirimkan oleh Sistem Bank/TP}}
  \end{itemize}

\end{itemize}